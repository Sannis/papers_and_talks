\cleardoublepage
\phantomsection
\section*{Введение}
\addcontentsline{toc}{section}{Введение}


\renewcommand{\theequation}{В.\arabic{equation}}


Явление самофокусировки стало известно ещё в 1960-х годах \cite{BespalovTalanov1966}. Её наблюдали при~фокусировке
импульсного лазерного излучения в органические жидкости. Теоретические исследования в этом направлении
были предприняты Г. А. Аскарьяном в~1962~г.~\cite{Askaryan1962}, им же было определено, что
самофокусировка возникает лишь в пучках, мощность которых больше некоторого критического
значения, определяемого свойствами среды. Сфокусированный за счёт нелинейного изменения показателя преломления
пучок продолжает распространяться без разфокусировки, что принципиально отличается от обычной
фокусировки с помощью линз. При этом образуется плазма, свечение которой наблюдалось
в эксперименте~\cite{PilipetskiyRustamov1965} и именно подобные светящиеся нити называются филаментами.


Самофокусировка и последующая филаментация лазерных импульсов порождает множество
физических процессов, имеющих как фундаментально-научное, так и чисто прикладное значение. В филаменте
пространственно локализуется энергия импульса, возникает плазменный канал, возникновение
филамента обычно сопровождается конической эмиссией, формированием излучения
суперконтинуума~\cite{KandidovShlenovKosarevaReview2009}. Это даёт новые возможности
для применения лазеров в атмосферной оптике, микрооптике и некоторых других областях.
Этой тематике посвящено много работ, в том числе и опирающихся на численный эксперимент.
В настоящее время наиболее активные исследования идут в области многофиламентации,
атмосферного зондирования на основе генерируемого филаментом суперконтинуума и FIBS-спектроскопии
(filament induced breakdown spectroscopy), управления формированием и распространением филамента.


Рассмотрим механизм возникновения филаментов. Из-за небольших флуктуаций показателя преломления
в среде на поперечном профиле возникают <<горячие точки>> с~большей интенсивностью, из которых под действием керровской самофокусировки
возникают отдельные пучки. При их взаимодействии со средой образуется плазма, которая дефокусирует эти пучки
и не даёт интенсивности неограниченно возрастать. \mbox{Если пиковая} мощность лазерного импульса в десятки или даже сотни раз превышает критическую,
то возникает множество таких филаментов.  Вокруг филаментов возникает область поля, называемая энергетическим резервуаром,
в которой содержится большая часть энергии исходного импульса.Интерференция излучения вокруг отдельных филаментов приводит к локальному превышению
критической мощности и возникновению новых <<горячих точек>>. Этот~процесс подробно описан в \cite{CenturionPuTsangPsaltis2005}.
Процесс образования <<горячих точек>> носит случайный характер и трудно поддаётся контролю, тем самым увеличивая
важность исследований в этой области. Это является принципиально важным для спектроскопии
FIBS~\cite{RohwetterMejeanStelmaszczyk2004}, так как необходимо, чтобы филамент возник в непосредственной
близости от поверхности мишени, спектр флюоресценции которой мы хотим измерить. Также точность в предсказании точки возникновения филамента вдоль оси распространения и поперечным координатам
важна для приложений в~области микромодификации оптических материалов.


Возможны два типа управления филаментацией: управление во времени, осуществляемоге
за счёт изменения длительности и~начальной модуляции фазы импульса, и~управление в пространстве,
при котором меняются фокусировка, распределения интенсивности и фаза излучения в
поперечном сечении пучка.


Влияние начальной фазовой модуляции на образование филамента определяется двумя факторами.
Первый из них состоит в уменьшении начальной мощности и не зависит от~знака фазовой
модуляции. Длительность импульса при фазовой модуляции увеличивается, пиковая мощность
уменьшается и в соответствии с формулой Марбургера~(\ref{ModelMarburgerDim}) увеличивается расстояние до зарождения
филамента. Второй фактор заключается в предварительной компенсации дисперсии групповых скоростей
и~зависит от~знака фазовой модуляции. В среде с нормальной дисперсией импульс с отрицательной
фазовой модуляцией сжимается, его пиковая мощность увеличивается при распространении и,
следовательно, расстояние филаментации уменьшается.


За счёт перераспределения интенсивности в поперечном сечении пучка возможно крупномасштабное
управление филаментацией. Так, можно изменять расходимость выходного пучка лазерной системы
или фокусировать~\cite{FibichSivanEhlrich2006} лазерный луч для смещения области зарождения филамента.


Можно управлять расстоянием филаментации и с помощью масштабирования пучка в поперечном сечении.
Так, увеличение отношений полуосей эллиптического распределения интенсивности приводит
к увеличению критической мощности и, как следствие, к увеличению
расстояния до зарождения филамента~\cite{KandidovFedorov2004}.


Применительно к случаям, когда мощность импульса составляет сотни и более критических
мощностей самофокусировки, развиваются методы создания упорядоченных структур филаментов.
Это можно делать за счёт наложения масок с отверстиями. Если пропустить пучок через
диафрагму для формирования супергауссова профиля импульса, то при многократном превышении критической
мощности на границе импульса формируется нерегулярное кольцо филаментов~\cite{GrowIshaayaVuong2006}.
В~\cite{PanovKosarevaKandidov2006}~численно показано, что создание периодической структуры
подавляет случайный процесс возникновения филаментации, увеличивая предсказуемость его развития.
Таким образом, за счёт создания предварительного распределения интенсивности в поперечном
сечении пучка возможно улучшить качество прогнозирования возникновения филаментов в случае распространения
в условиях флуктуаций показателя преломления на трассе вследствии атмосферной турбулентности или наличия аэрозолей.


Однако подобный метод уменьшает мощность выходного пучка лазерной системы по~сравнению
с мощностью используемого в ней лазера. Поэтому для сильных полей предпочтительной
является фазовая модуляция пучка. Как правило, для этого используют или фазовые маски, или
массивы микролинз.


В \cite{TingTing2006} показано, что при распространении двух импульсов с различной начальной фазой
характер взаимодействия между ними определяется разностью фаз между филаментами.
В случае, если разница фаз невелика ($\varphi < 0.5\pi$), филаменты притягивались друг к другу
и со временем сливались в один. Чем больше фазовый сдвиг, тем более нестабильным оказывается его поведение.
Также в этой работе был проведён расчёт взаимодействия двух импульсов, распространяющихся под небольшим углом друг к другу.
Если этот угол невелик ($\theta = 0.01^{\circ}$) и импульсы синфазны, то образовывается филамент, распространяющийся по среднему направлению.
В случае, если импульсы имеют разность фаз $\varphi = \pi$, то импульсы деструктивно взаимодействуют и филамент не возникает.
Если же угол между направлениями распространения импульсов больше ($\theta = 0.1^{\circ}$),
то зависимость характера взаимодействия от разницы фаз становится менее выраженной
и после прохождения точки пересечения своих осей импульсы продолжают независимое распространение.
Было показано, что характер взаимодействия зависит также от соотношения расстояния между филаментами и~диаметра их энергетических резервуаров:
именно их перекрытием и~вызвано притяжение или отталкивание филаментов.
По-видимому, похожие эффекты можно наблюдать и~в~случае большего количества импульсов,
либо при разнице фаз в соседних элементах пучка с~регулярной поперечной структурой.
Исследованию особенностей в~процессе самофокусировки пучков с резулярной поперечной структурой
посвящена гл.~\ref{sec:beams} настоящей работы.


В большинстве статей по филаментации используются экспериментальные данные,
полученные для Ti:Sapphire лазера с длинной волны излучения 800 нм и меньше.
Временами исследования проводятся для импульсов на длине волны 1240 нм. Это накладывает определённые ограничения
на параметры филаментов, которые можно получить. Как показывают аналитические оценки, с увеличением длины волны
возможно увеличение пиковой интенсивности в филаменте, уменьшение концентрации плазмы
и увеличение поперечных размеров филамента, что может быть полезно для различных приложений.


Изучению аналитической зависимости критической мощности самофокусировки для различных длин волн
и параметров филаментов в воздухе посвящена работа~\cite{FedorovKandidovDifferentWavelengths2008}.
В \cite{FedorovKandidovAirN22008} приводятся данные о том, что нелинейный показатель преломления $n_2$,
характеризующий зависимость показателя преломления от интесивности, зависит от длины волны:

\begin{equation*}
n(I) = n_0 + n_2 I,
\end{equation*}

\begin{equation}\label{OverviewN2}
n_2(\lambda) = 3 + \dfrac{6.37 \cdot 10^5}{\lambda^2}.
\end{equation}

Это позволило автору, подставляя выражение для $n_2(\lambda)$ в формулу для критической мощности
и раскладывая её в ряд Тейлора получить простую аппроксимацию зависимости $P_{cr}(\lambda)$:

\begin{equation}\label{OverviewP_cr}
P_{cr}(\lambda) = R_{cr}\dfrac{\lambda^2}{8 \pi n_0 n_2(\lambda)},
\end{equation}

\begin{equation}\label{OverviewP_cr_Fedorov}
P_{cr}(\lambda) = P_{cr}(\lambda^{800})(1 + 3.12 \cdot 10^{-3}\Delta\lambda + 2.36 \cdot 10^{-6}\Delta\lambda^2 - 1.82 \cdot 10^{-10}\Delta\lambda^3),
\end{equation}

\noindent где $R_{cr}$ зависит от формы импульса, $P_{cr}(\lambda^{800}) = 2.4$ ГВт для воздуха,
а $\Delta\lambda = (\lambda - \lambda^{800})$ измеряется в нанометрах.
В работе приведены рассчитанные параметры филамента
(расстояние филаментации, пиковая интенсивность, концентрация электронов, радиусы филамента и плазменного канала)
для импульса с одинаковой энергией при длинах волн $\lambda = 248, 400, 600, 800, 1060$ и $1240$~нм
и~эти~данные хорошо согласуются с эксперименальными работами.

\begin{table}[H]
\begin{tabular*}{\textwidth}{@{\extracolsep{\fill}} |r|c|c|c|c|c|c|}
\hline
$\lambda$, нм & $P/P_{cr}$ & $z_{fil}$, м & $I_{fil}$,  $10^{13}$, $\textrm{Вт}/\textrm{см}^2$ & $N_e^{fil}$, $10^{16} \textrm{см}^{-3}$ & $r_{fil}$, $\textrm{мкм}$ & $r_{pl}$, $\textrm{мкм}$ \\
\hline
248  &  163.95 &  1.52 &          1.13 &            10.89 &      71.30 &    14.54 \\
400  &   32.95 &  1.69 &          4.83 &             9.66 &      66.09 &    19.03 \\
600  &   10.01 &  2.34 &          5.93 &             3.54 &      88.52 &    26.16 \\
800  &    4.72 &  3.04 &          6.54 &             1.73 &     112.38 &    32.69 \\
1060 &    2.44 &  4.23 &          6.66 &             0.82 &     145.44 &    39.63 \\
1240 &    1.68 &  5.79 &          6.49 &             0.48 &     171.69 &    45.02 \\
\hline
\end{tabular*}
\\[1ex]
\caption{Характеристики филамента и плазменного канала, полученные в \cite{FedorovKandidovDifferentWavelengths2008}. \\
         Параметры импульса: $W = 8$~мДж, $\tau_0 = 100$~фс, $a_0 = 1.2$~мм, $I_0 = 10^{12} \textrm{Вт}/\textrm{см}^2$.}
\label{tab:OverviewFedorovData}
\end{table}


Эти данные ограничены сверху длиной волны 1240 нм.
Однако, уже не первое десятилетие идут исследования, направленные на создания мощных лазеров в среднем ИК"=диапазоне.
Одна из областей применения подобных лазеров "--- эксперименты по ускорению частиц. За счёт
квадратичной зависимости пондеромоторных сил от длины волны при одинаковой мощности
лазер на длине волны около 10 мкм позволяет получать ускорение на два порядка большее, чем лазер ближнего ИК"=диапазона.
Одним из пионеров в этой области является Брукхэвенская национальная лаборатория (BNL) в США \cite{BrookhavenCO2},
где в 1995 году был построен $CO_2$-лазер, генерирующий импульсы мощностью до 20 ГВт при длительности 50 пикосекунд.


Одной из проблем, стоящих перед разработчиками мощных лазеров "--- необходимость генерировать очень короткие импульсы,
которые имеют широкий спектр. Для~этого необходимо, чтобы рабочая среда имела очень широкий спектр усиления.
Это достигается за счёт создания большого давления газа активной среды, обычно от нескольких до десятка атмосфер.
Также возможен кропотливый подбор концентраций газов для~усилителя (кроме $CO_2$, в нём обычно присутствует азот и гелий, иногда водород).
Усиление происходит за счёт энергетических переходов между молекулами $CO_2$ и возможно изменять их вероятности за счёт использования в смеси
молекул, состоящих из различных изотопов углерода и кислорода. Так, в \cite{PicosecondAmplifier2007} приведено описание усилителя,
в котором использовалась смесь из пяти эквивалентных молекул, что позволяло добиться пятикратного усиления импульсов
на длинах волн $9.6$ и $10.4$ микрон, характерных для $CO_2$ лазера ИК"=диапазона. В этой же статье говорится о теоретической возможности сжимать полученные
импульсы до длительности 500 фемтосекунд.


Использование описанного выше усилителя наравне с собственными разработками позволило создать в Санкт-Петербурге
установку по генерации лазерных импульсов тераватной мощности длительностью до 15 пикосекунд \cite{TerawattCO2laser2006}.
В статье указывается на~неопубликованные материалы по уменьшению длительности импульсов до 6 пс
и возлагаются большие надежды на дальнейшее развитие этой техники в области фемтосекундных длительностей и петаваттных мощностей.


Тем, кто глубоко интересуется данной тематикой, будет интересна статья \cite{BaranovKuchinskiyTomashevich2008},
где~описывается создание широкоапертурного $CO_2$-усилителя сверхатмосферного давления, позволяющего получать
импульсы мощностью $\simeq 100$ ГВт и длительностью $\simeq 100$ пс. В статье даются ценные указания по конструкции усилителя
и решению технических проблем, возникающих при его создании.


Согласно последним данным \cite{HaberbergerTochitskyJoshi2010}, ракордом мощности можно считать импульсы мощностью
15 ТВт и длительностью 3 пикосекунды, полученные в августе этого года. Эта установка генерирует цуг импульсов
с периодом 18 пикосекунд с различной энергией импульсов от 10 до 45 Дж.


Рассмотрим вопрос возможности филаментации лазерного излучения с длиной волны около 10 мкм. Согласно грубой оценке
$P_{cr} \sim \lambda^{-2}$ или более точной оценке (\ref{OverviewP_cr})-(\ref{OverviewP_cr_Fedorov}), можно ожидать
критическую мощность самофокусировки для такого излучения в районе 350--650 ГВт. Как показано выше, генерация пикосекундных
импульсов такой мощности вполне возможна. А также есть теоретические предпосылки для создания лазерных систем
с тераваттной мощностью и длительностью импульсов до 500 фемтосекунд. Таким образом, должна быть возможность наблюдать
филаментацию на длине волны 10 мкм в воздухе.


\cleardoublepage
\phantomsection
\subsection*{Цели работы}
\addcontentsline{toc}{section}{Цели работы}

В дипломной работе рассмотрены две задачи о филаментации лазерного излучения в воздухе.


Первая задача связана с проблемой управления началом филаментации. В ней анализируется
множественная филаментация лазерных импульсов с регулярной поперечной структурой.
Рассматривается стационарная самофокусировка. Анализ включает в себя исследование различных режимов
филаментации в зависимости от мощности пучка и разности фаз в соседних элементах структуры.
Также исследуется влияние амплитудного шума в исходном пучке на этот процесс.


Целью второй части дипломной работы является анализ параметров филаментов на длине волны излучения
$CO_2$-лазера. Рассмотрена нестационарная задача с учётом нелинейной ионизации молекул воздуха
и выполнен сравнительный анализ филаментации на длинах волн 0.8 и 10 мкм.


Часть расчётов проводилась на суперкомпьютере СКИФ МГУ <<Чебышёв>> и вычислительном кластере МЛЦ МГУ.
В связи с этим одной из методических целей работы стал сравнительный анализ параллельных численных
алгоритмов решения уравнения квазиоптики. В реальных условиях измерена скорость выполнения алгоритмов
для метода Фурье преобразования и метода на основе разностных схем для различных размеров рассчётной сетки
и количества задействованных процессоров. Полученные результаты и описание алгоритмов приведены
в приложении к дипломной работе.


\renewcommand{\theequation}{\arabic{section}.\arabic{subsection}.\arabic{equation}}

