\section{Метод Фурье.}


\subsection{Освоение \fftw.}

\begin{itemize}
	\item Разобраться с подключением {\fftw} в проект Visual Studio. Мануал по {\fftw} 2.1.3 есть на сайте \href{http://fftw.org}{fftw.org}.
	\item Разобраться с распределением матрицы между процессами, необходимым для \fftw.
	\item Разобраться с расположением результата одного фурье-преобразования и нормировкой.
	\item Реализовать \code{fftw\_mpi(...)} туда-обратно при гауссовых начальных условиях в двумерном случае.
\end{itemize}


\subsection{Дифракция в линейной среде.}
\begin{multline*}
	\left\{
	\begin{array}{rcl}
		2i\dfrac{\partial E}{\partial{z}}&=&\dfrac{\partial^2 E}{\partial{x^2}}+
		\dfrac{\partial^2 E}{\partial{y^2}}\\
		E(x,y,0)&=&\exp\left\{-\dfrac{x^2+y^2}{2}\right\},\quad (x,y)\in[-l,l]^2
	\end{array}
	\right.
	\quad{\hbox to 0.7cm{\rightarrowfill}}\\
	{\hbox to 0.7cm{\rightarrowfill}}\quad
	\Biggl[
	E(x,y,z)=\sum\limits_{j,k}\tilde{E}_{jk}(z)\exp\left\{\dfrac{2\pi ijx}{N}\right\}
	\exp\left\{\dfrac{2\pi iky}{N}\right\}
	\Biggr]
	\quad{\hbox to 0.7cm{\rightarrowfill}}\\
	{\hbox to 0.7cm{\rightarrowfill}}\quad
	\left\{
	\begin{array}{rcl}
		2i\dfrac{\partial\tilde E}{\partial{z}}&=&\left(\dfrac{2\pi i}{N}\right)^2(j^2+k^2)
		\tilde E(z)\\
		\tilde E_{jk}(0)&=&\tilde E^{(0)}
	\end{array}
	\right.
	\quad{\hbox to 0.7cm{\rightarrowfill}}\\
	{\hbox to 0.7cm{\rightarrowfill}}\quad
	\tilde E_{jk}(z)=\tilde E^{(0)}\exp\left\{i\dfrac{2\pi^2}{N^2}(j^2+k^2)z\right\}
\end{multline*}

\begin{multline*}
	\text{Итого:}\hspace{0.5cm}E(x,y,0)
	\hspace{0.5cm}{\buildrel\hbox{2D FFT}\over{\hbox to 2cm{\rightarrowfill}}}
	\hspace{0.5cm}\tilde E_{jk}(0)
	\hspace{0.5cm}\hbox to 1cm{\rightarrowfill}\\
	\hbox to 1cm{\rightarrowfill}
	\hspace{0.5cm}\tilde E_{jk}(0)\exp\left\{i\dfrac{2\pi^2}{N^2}(j^2+k^2)z\right\}
	\hspace{0.5cm}{\buildrel{\hbox{2D FFT}^{-1}}\over{\hbox to 2cm{\rightarrowfill}}}
	\hspace{0.5cm}E(x,y,z)
\end{multline*}

\begin{itemize}
	\item Реализовать параллельный расчет дифракции гауссового пучка.
        \item Сравнить с аналитическим решением.
        \item Исследовать зависимость времени счёта от следующих параметров:
		\begin{itemize}
			\item с буфером и без него;
			\item \code{FFTW\_NORMAL\_ORDER} vs \code{FFTW\_TRANSPOSED\_ORDER};
			\item (опционально) \code{FFTW\_ESTIMATE} vs \code{FFTW\_MEASURE};
			\item (опционально) \code{wisdom}, OpenMP, ...
		\end{itemize}
	\item Выбрать оптимальные параметры для использования в следующем пункте.
\end{itemize}


\subsection{Дифракция в нелинейной среде.}
\begin{equation*}
	\left\{
	\begin{array}{rcl}
		2i\dfrac{\partial E}{\partial z}&=&R\left|E\right|^2E\\
		E(x,y,0)&=&E_0(x,y)
	\end{array}
	\right.
	\rightarrow
	E(x,y,z)=E_0(x,y)\exp\left\{-i\dfrac{R\left|E_0(x,y)\right|^2}{2}z\right\}
\end{equation*}

\begin{itemize}
	\item Выполнять шаг дифракции и шаг нелинейности в разном порядке:
	\begin{enumerate}
		\item <<дифракция "--- нелинейность>>
		\item <<нелинейность "--- дифракция>>
		\item чередовать
	\end{enumerate}
	\item Изменять шаг интегрирования для выполнения условия $\Delta\varphi_{\text{нл}}=\dfrac{R\left|E_{max}\right|^2}{2}\Delta z < 0.1$.
	\item Использовать оптимальные параметры \fftw, найденные в предыдущем пункте.
	\item Получить формулу Марбургера для гауссова пучка и ее аналог для $\ch^{-1}$ пучка.
\end{itemize}

