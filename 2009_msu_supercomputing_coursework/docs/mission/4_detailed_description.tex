\section{Входные параметры и формат выходных файлов.}
\label{sec:detailed_description}


\subsection{Начальные условия.}
Начальное распределение должно иметь вид гауссова пучка. Размер счётной области -- 10 радиусов пучкая.
Сгенерировать такое распределение можно с помощью \href{http://github.com/Sannis/create_2d_func/}{create\_2d\_func}
(\href{http://github.com/Sannis/create_2d_func/tarball/v1.0}{стабильная версия 1.0}) со следующими аргументами:
\begin{verbatim}
$> mpirun -np 8 ./create_2d_func -n 1024 -f gauss \
       -l 5 --a0 1 --r0 1 ./gauss_n1024_l5.cpl
\end{verbatim}


\subsection{Проводимые расчёты.}
\begin{enumerate}
	\item Время выполнения одного шага дифракции с использованием различных флагов \fftw. Параметры:
		\begin{itemize}
			\item $N = 1024, 8192$.
			\item $np = 1, 8, 32$ для СКИФ <<Чебышёв>> \\ и $np = 128, 1024$ для IBM Bluegene/P.
		\end{itemize}
		Сравнить скорость работы {\fftw} при использовании дополнительного буфера и без него,
		степень зависимости времени выполнения от использования флагов \\ FFTW\_NORMAL\_ORDER и FFTW\_TRANSPOSED\_ORDER.
	\item Время расчёта распространения пучка в нелинейной среде на одну дифракционную длину(до $z=1$). Параметры:
		\begin{itemize}
			\item $R = 5,\text{ }\Delta\varphi < 0.1$.
			\item $N = 512, 1024, 2048, 4096, 8192, 16384, 32768$.
			\item $np = 1, 2, 4, 8, 16, 32, 64, 128$ для СКИФ <<Чебышёв>> \\ и $np = 128, 256, 512, 1024$ для IBM Bluegene/P.
		\end{itemize}
		Провести замер времени выполнения расчётной части программы в друх вариантах:
		только распространение и распространение с сохранением результируещего поля и максимальной интенсивности пучка каждые 10 шагов.
		После расчёта с сохранением бинарные данные можно удалить или использовать для оценки точности метода, чтобы сократить
		число запусков программы. Столбец полной можности пучка в выходной таблице можно заполнить нулями.
		
	\item Точность алгоритмов для случая линейного распространения на одну дифракционную длину(до $z=1$). Параметры:
		\begin{itemize}
			\item $R = 0$.
			\item $N = 512, 2048, 8192$.
			\item $\Delta z = 0.001$, соответственно 1000 шагов.
		\end{itemize}
		Результатом работы программы должна быть таблица с данными об изменении максимальной интенсивности
		и полной мощности пучка при распространении. Формат таблицы будет приведён ниже.
		Также необходимо сохранить в файл конечное распределение поля при $z=1$.
	
	\item Точность алгоритмов для случая нелинейного распространения на одну дифракционную длину(до $z=1$). Параметры:
		\begin{itemize}
			\item $R = 5,\text{ }\Delta\varphi < 0.1$.
			\item $N = 512, 2048, 8192$.
		\end{itemize}
		Результатом работы программы должна быть таблица с данными об изменении максимальной интенсивности
		и полной мощности пучка при распространении. Формат таблицы будет приведён ниже.
		Также необходимо сохранить в файл конечное распределение поля при $z=1$.
\end{enumerate}


\subsection{Правила именования файлов и папок с результатами.}
Результаты должны располагаться в папках следующего вида(относительно папки запускаемой программы):
\begin{itemize}
	\item Для сравнения скорости работы {\fftw} с разными флагами: \\
		\texttt{./results/Skif/FFTW\_compare/(no|with)buffer\_(no|with)transpose/N1024\_NP16=4x4/}
	\item Для замеров времени без сохранения: \\
		\texttt{./results/Skif/Time\_no\_save/N1024\_NP16=4x4/}
	\item Для замеров времени c сохранением: \\
		\texttt{./results/Skif/Time\_with\_save/N1024\_NP16=4x4/}
	\item Для оценки точности: \\
		\texttt{./results/Skif/Accuracy\_r0/N1024\_NP16=4x4/} и \\
		\texttt{./results/Skif/Accuracy\_r5/N1024\_NP16=4x4/}
\end{itemize}
В каждой папке должен располагаться файл log.txt с входными данными и результатами работы программы. Бинарные файлы с распределением поля на отдельных шагах должны располагаться в файлах вида out\_00070.cpl. Распределение поля в конце трассы (при $z=1$) должно быть сохранено под именем out\_z1.cpl.


\subsection{Формат выходного лога программы.}
В начале вывода программы должны присутствовать значения параметров сетки и расчётных параметров.
Далее должна следовать таблица, содержащая колонки со значениями текущего шага, координаты, максимальной интенсивности и полной мощности пучка. В последней строке нужно вывести время выполнения расчётной части программы. \\
Образец:
\begin{verbatim}
    =========================================
    === Propagation: Diffraction and Kerr ===
    =========================================

MPI grid size: 1 (1x1)

N: 256
L: 5.000000
Impulse file: ../../data/gauss_n256_l5.cpl

R: 0.000000
dz: 0.001000 (dphi < 0.1)
Steps: 1000

n        dz          z           I_max(z)          P(z)
00000    0.010000    0.000000    1.000000000000    0.999999999997
00001    0.010000    0.010000    0.999900162406    0.999999999997
00002    0.010000    0.020000    0.999600769015    0.999999999997

...

Execution time(sec):
144.044038
\end{verbatim}


\subsection{Обработка результатов.}
\begin{itemize}
	\item Для оценки точности алгоритмов необходимо произвести сравнение формы импульсов после распространения на одинаковую длину.
		В случае линейного распространения результат работы каждого алгоритма можно сравнить с аналитическим решением,
		в случае нелинейного --- только между собой. Для сравнения используется программа \href{http://github.com/Sannis/bindiff/}{bindiff}
(\href{http://github.com/Sannis/bindiff/tarball/v1.0}{стабильная версия 0.1}).
		
	\item Построение графиков осуществить в автоматическом режиме с использованием bash скриптов и \href{http://www.gnuplot.info/}{gnuplot}.
	
	\item Визуализировать бинарные данные можно с использованием программы \href{http://github.com/Sannis/bin2gif/}{bin2gif}
(\href{http://github.com/Sannis/bin2gif/tarball/v0.3}{стабильная версия 0.3}).
\end{itemize}


\subsection{Отчётные данные.}

\begin{itemize}
	\item Времена работы алгоритма {\fftw} при различных параметрах. Выбранные оптимальные параметры.
	\item Времена работы всех перечисленных алгоритмов без сохранения и с учётом сохранения. Ускорение работы программ в зависимости от количества процессов.
	\item Интегральная среднеквадратичная ошибка в распределении поля при дифракции гауссова пучка (линейный случай).
	\item \ZZZ{Интегральная среднеквадратичная ошибка в соответствии с формулой Марбургера для расстояния филаментации (нелинейный случай).}
\end{itemize}
