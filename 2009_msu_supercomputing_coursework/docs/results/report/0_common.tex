\section{Введение}

%Ориентировочный план отчета:
%\begin{enumerate}
%    \item Введение.
%    \item Постановка задачи.
%    \item Параллельные алгоритмы решения задачи.
%    \begin{itemize}
%        \item Использование Фурье-метода.
%        \item Использование явной схемы.
%        \item Использование неявной схемы.
%    \end{itemize}
%    \item Сравнение алгоритмов.
%    \item Выводы.
%    \item Благодарности.
%    \item Список литературы.
%\end{enumerate}

Распространение мощных фемтосекундных лазерных импульсов в атмосфере сопровождается явлением филаментации, когда существенная часть энергии излучения концентрируется в <<горячих точках>> в поперечном сечении импульса, которые наблюдаются в виде тонких протяженных нитей.
Впервые светящаяся нить, или филамент, была зарегистрирована в 1965 году при фокусировке наносекундных лазерных импульсов в кювету с органическими жидкостями \cite{FirstFilament}.
С созданием фемтосекундных лазерных установок стало возможным получение протяженных (до нескольких сотен метров) филаментов при распространении излучения в газовых средах, в частности, в атмосфере.
Краткий обзор этих и других экспериментальных результатов дан в \cite{KandidovShlenovKosarevaReview2009}.

Причиной начала формирования нитей-филаментов является эффект Керра, вызывающий самофокусировку пучка в среде.
Самофокусировка имеет место, если мощность пучка превосходит некоторую критическую $P_{cr}$, зависящую как от параметров среды (линейного и нелинейного коэффициентов преломления), так и от параметров пучка (таких как длина волны излучения и радиус, а также от формы пучка, степени его осесимметричности и так далее).
С простейшей теорией самофокусировки можно ознакомиться в \cite{AkhmanovNikitin}.
Рост интенсивности останавливается в нелинейном фокусе за счет дефокусирующего действия наведенной излучением плазмы, возникающей в результате действия нескольких механизмов, прежде всего многофотонной и туннельной ионизации.
Обычно филамент формируется на оси пучка на расстоянии, соответствующем положению нелинейного фокуса для центрального, наиболее мощного временного слоя.
На его положение могут оказывать влияние такие факторы, как флуктуации показателя преломления в турбулентной атмосфере и фазовые флуктуации в начальном пучке.

\section{Постановка задачи}

В простейшем случае уравнение, описывающее самофокусировку гауссового пучка, записывается в виде:
\begin{equation}\label{MainDim}
    \left\{
	\begin{array}{rcl}
		2ik\dfrac{\partial E}{\partial z}&=&\dfrac{\partial^2 E}{\partial x^2}+
		\dfrac{\partial^2 E}{\partial y^2} + \dfrac{2k^2}{n_0}n_2\left|E\left(x,y,z\right)\right|^2E\left(x,y,z\right)\\
        \\
		E(x,y,0)&=&E_0\exp\left\{-\dfrac{x^2+y^2}{2a_0^2}\right\},\quad (x,y)\in[-l,l]^2
	\end{array}
	\right.
\end{equation}

Здесь $E\left(x,y,z\right)$ "--- напряженность электрического поля, $k$ "--- волновое число, $z$ "--- координата вдоль оси распространения лазерного пучка, $x,y$ "--- координаты в поперечном сечении, $n_0$ и $n_2$ "--- линейный и нелинейный коэффициент преломления.
В данном уравнении учтены такие физические факторы, как дифракция лазерного пучка и кубическая (по полю) нелинейность.
Таким образом, оно описывает начальный этап филаментации, а именно распространение пучка до нелинейного фокуса, когда возросшая интенсивность приведет к плазмообразованию и дефокусировке.
Эти процессы требуют введения дополнительных слагаемых в правую часть (\ref{MainDim}).
Однако численное решение нелинейного уравнения квазиоптики представляет и самостоятельный интерес.

После обезразмеривания $E=\tilde{E}\cdot E_0$, $x=\tilde{x}\cdot a_0$, $y=\tilde{y}\cdot a_0$, $z=\tilde{z}\cdot ka_0^2$ система будет выглядеть следующим образом:
\begin{equation}\label{MainNoDim}
    \left\{
	\begin{array}{rcl}
		2i\dfrac{\partial \tilde{E}}{\partial \tilde{z}}&=&\Delta_{\perp}\tilde{E} + R\left|\tilde{E}\right|^2\tilde{E}\\
        \\		\tilde{E}(\tilde{x},\tilde{y},0)&= &\exp\left\{-\dfrac{\tilde{x}^2+\tilde{y}^2}{2}\right\}, \quad (x,y)\in\left[-\dfrac{l}{a_0},\dfrac{l}{a_0}\right]^2
	\end{array}
	\right.
\end{equation}

Основные проблемы численного моделирования задачи филаментации лазерных импульсов связаны с многомасшабностью задачи.
Поперечные масштабы пучка примерно на два порядка превосходят возникающие в нем структуры.
В то же время размер расчетной сетки должен на порядок превосходить радиус пучка, чтобы границы сетки не отсекали существенные части пучка, а также, чтобы иметь некоторую <<буферную область>>, в которую могла бы расширяться низкоинтенсивная периферийная часть пучка.
В противном случае неизбежно возникновение краевых эффектов, приводящих к искажению решения.
Кроме того, на диаметр филамента должно приходиться достаточное количество точек (не менее 10), иначе резкие перепады интенсивности в окрестности филамента будут содержать слишком высокие пространственные частоты, что приведет к невыполнению критерия Найквиста, наложению частот и, как следствие, неадекватности получаемого решения.

Таким образом, количество точек в поперечном сечении достигает $10^4$ по каждой поперечной координате. Если учесть теперь, что количество временных слоев составляет величину порядка $10^2-10^3$, то общее количество точек достигает величины $10^{11}$, а потребность в оперативной памяти "--- величины порядка 100 Гб.

\section{Параллельные алгоритмы решения задачи}\label{SplitMethod}

Были предложены три метода численного решения поставленной задачи.
Один метод основан на использовании явной схемы для уравнения (\ref{MainNoDim}).
Два других предусматривают предварительное расщепление по физическим факторам.
Интегрирование нелинейного уравнения дифракции сводится к последовательному интегрированию на каждом шаге интегрирования двух уравнений: первое описывает только дифракцию, второе "--- только нелинейность.

Эти уравнения имеют вид:
\begin{equation}\label{Split}
    \left\{
	\begin{array}{rcl}
		2i\dfrac{\partial E}{\partial z}&=&\Delta_{\perp}E \\
        \\
        2i\dfrac{\partial E}{\partial z}&=&R\left|E\right|^2E
	\end{array}
	\right.
\end{equation}

Интегрирование уравнения для нелинейности не представляет проблем:
\begin{equation}\label{KerrSolution}
    E(x,y,z_i + \Delta z) = E(x,y,z_i)\exp\left(-\frac{iR}{2}\left|E(x,y,z_i)\right|^2\Delta z\right)
\end{equation}

Отметим, что поскольку нелинейность считается локальной, то есть набег фазы в точке поперечного сечения зависит только от значения интенсивности поля в этой же точки, поэтому применение метода геометрического параллелизма не вызывает проблем.

Для интегрирование уравнения дифракции была использована неявная численная схема и метод, использующий быстрое преобразование Фурье.

Шаг интегрирования по оси $z$ по мере приближения к нелинейному фокусу должен уменьшаться, поскольку рост интенсивности при самофокусировке приводит к росту нелинейного фазового набега засчет эффекта Керра.
Критерием пригодности шага был следующим:
\begin{equation}\label{StepCriterium}
    \max\limits_{x,y}\dfrac{R}{2}\left|E(z_i)\right|^2\Delta z \leqslant 0{,}01.
\end{equation}

