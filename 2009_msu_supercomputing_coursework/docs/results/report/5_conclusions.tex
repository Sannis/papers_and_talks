\section{Выводы}

В рамках проектной работы по курсу <<Суперкомпьютерные технологии>> нами было проведено исследование
работы различных алгоритмов решения нелинейного уравнения квазиоптики, известного также как
нелинейное уравнение Шредингера. Рассматривались методы с использованием разностных схем и гибридный метод с расщепление по физическим факторам, где нелинейность учитывалась локально, а для решения
линейной части уравнения использовался метод с применением преобразования Фурье.


Тестовые эксперименты показали, что при использовании схемы предиктор-корректор (Рунге-Кутта 4-го порядка) из-за наличия поперечных координат и участия в уравнении производной по ним необходимо использовать довольно маленький шаг по координате z для получения устойчивого решения. Таким образом, эту схему имеет смысл использовать в расчётах,
где величина шага уже лимитирована другими особенностями поставленной задачи. Также следует отметить, что для реализации этого алгоритма необходимо держать в памяти 3 временные матрицы равные по размеру основной, а на каждом шаге необходимо 4 раза обменяться граничными значениями локальных блоков матрицы.


В случае применения неявной консервативной схемы, устойчивой при любом шаге интегрирования, шаг необходимо выбирать
основываясь на эмпирических физических оценках. Кроме того, для реализации этого алгоритма требуется только одна дополнительная матрица, равная основной, для хранения прогоночных коэффициентов. Алгоритм показал отличную масштабируемость на кластера IBM Bluegene/P, которая не сильно пострадала даже в случае периодического сохранения данных вычислений на диск. В основном это объясняется небольшой тактовой частотой процессоров при наличии быстрой сети и применения MPI Parallel I/O.


При использовании Фурье-метода можно обойтись без использования дополнительной матрицы, тем самым по сравнению с методом Рунге-Кутта увеличить возможной расчётной сетки в 2 раза по каждой координате. Из-за более сложной организации пересылок при расчёте параллельного Фурье-преобразования по сравнению с остальными алгоритмами его масштабируемость ниже. Однако скорость вычислений для этого метода больше, по крайней мере при использовании до 64 процессов, что для применения на СКИФ МГУ <<Чебышёв>> делает его более удачным.


Таким образом, нельзя чётко сказать, какой метод является лучше в общем случае. В случае, если задача не накладывает каких-то особых ограничений, лучше использовать Фурье-метод. Если же имеется возможность использовать для расчёта очень большое число процессоров, то возможно применения метода с неявной разностной схемой. Также этот метод будет применим для задач с неравномерной сеткой по поперечному сечению, так как для них нет метода быстрого преобразования Фурье и этот метод теряет свою актуальность. 