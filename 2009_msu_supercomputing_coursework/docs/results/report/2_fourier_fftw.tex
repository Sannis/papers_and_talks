\subsection{Метод Фурье}

Метод Фурье основан на переход к двумерному Фурье-образу матрицы поля:
\begin{equation}\label{FourierDef}
    E\left(k_x, k_y, z\right)=\iint E\left(x,y,z\right)e^{-ik_xx-ik_yy}\,dxdy
\end{equation}

Дифракция в Фурье-представлении будет описываться следующим уравнением:
\begin{equation}\label{DiffractionFourier}
    2i\frac{\partial E\left(k_x, k_y, z\right)}{\partial z}= (-k_x^2-k_y^2)E\left(k_x, k_y, z\right),
\end{equation}
решение которого дается формулой
\begin{equation}\label{DiffractionFourierSolve}
    E\left(k_x, k_y, z+\Delta z\right)= E\left(k_x, k_y, z\right)\exp\left\{\dfrac{i}{2}(k_x^2+k_y^2)\left|E\left(k_x, k_y, z\right)\right|^2\right\}
\end{equation}

Процедура параллельного вычисления двумерного преобразования Фурье бралась из свободно распространяемой библиотеки FFTW \cite{fftw}.
Данная процедура предполагает ленточное распределение матрицы поля по процессам.
Кроме того, преобразование Фурье быстрее всего работает на матрицах, размеры которых являются степенями двойки.

Основными этапами выполнения преобразования являются создание плана, который содержит информацию об организации пересылок между процессами, и собственно выполнение преобразования.
Функция работает следующим образом.
Вначале выполняется Фурье-преобразование по строкам.
Этот этап происходит локально на каждом процессе, поскольку процесс содержит в своей оперативной памяти всю строку.
Затем происходит транспонирование распределенной матрицы, что связано с обменами данными между всеми процессами (то есть каждый обменивается с каждым).
Далее снова выполняется Фурье-преобразование по строкам, которые ранее были столбцами.

Существенной особенностью процедуры является расположение полученных коэффициентов в памяти процессоров.
Для преобразования Фурье естественным является транспонированное расположение результата в памяти всех процессов.
Этому соответствует ключ FFTW\_TRANSPOSED\_ORDER функции, реализующей преобразование Фурье.
Его альтернативой является ключ FFTW\_NORMAL\_ORDER, который после выполнения преобразования Фурье проводит дополнительно транспонирование матрицы.
Кроме того, один из параметров указанной функции может содержать дополнительный буферный массив для ускорения преобразования.
Наконец, при создании плана Фурье-преобразования существует возможность оптимизировать план с целью убыстрения работы функции.
Это достигается использованием ключей FFTW\_ESTIMATE (грубая оценка) и FFTW\_MEASURE (при этом производятся замеры времени пересылок, выполняемых в Фурье-преобразовании и их оптимизация).

Целью работы были замеры времени работы данного алгоритма в зависимости от размера матрицы, количества процессов, а также от комбинаций указанных ключей.
