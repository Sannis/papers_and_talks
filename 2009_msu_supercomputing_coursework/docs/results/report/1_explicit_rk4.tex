\subsection{Явная схема}

Уравнение (\ref{MainNoDim}) переписывается в виде:
\begin{equation}
	\Delta E(z_i)=E(z_i+\Delta z)-E(z_i)=\frac{\partial E}{\partial z}\Delta z =
    \frac{1}{2\textbf{i}}(\Delta_{\perp}E(z_i) + R |E(z_i)|^2 E) \Delta z = f(x, y, z)\Delta z
\end{equation}

Для его решения можно применять как очень нестабильный метод Эйлера,
так и заведомо более устойчивые методы из класса <<предиктор-корректор>>,
например, метод Рунге-Кутта 4-го порядка. Он состоит в последовательном вычислении
значения функции $f$ от некоторых аргументов и последующем усреднении полученного
таким образом значения приращения.
В случае, если у нас нет поперечных координат, это выглядело бы следующим образом:
\begin{equation}\label{rk4_method}
    \begin{aligned}
        k_1 & = f(E_i,t) \\
        k_2 & = f(E_i + \frac{k_1 \Delta z}{2}) \\
        k_3 & = f(E_i + \frac{k_2 \Delta z}{2}) \\
        k_4 & = f(E_i + k_3 \Delta z) \\
        E_{i+1} & = E_i + (k_1 + 2 k_2 + 2 k_3 + k_4) \Delta z /6
    \end{aligned}
\end{equation}

Соответственно, в нашем случае на каждом шаге нужно рассчитывать значение 4-х матриц $k_i[x_n,y_m]$.

Некоторые модификации данного метода используются другими исследователями этой проблемы. В целом же из-за зависимости функции $f()$ от поперечных координат(через поперечный лапласиан) делает эту схему неустойчивой для решения данной задачи. При этом из критерия Куранта-Фридрихса-Леви следует, что шаг по оси $z$ должен быть меньше чем $c(\Delta x)^{2}$, что при увеличении количества точек в поперечном сечении приводит к чрезвычайно малому шагу по $z$.
Однако это не ставит крест на алгоритме, так как он может применяться в тех случаях, когда требуется большая точность результатов и шаг заведомо должен быть довольно мал. 