\vspace{1em}
\textbf{3.2. Метод на основе преобразования Фурье.}
\vspace{0.5em}

<<Метод Фурье>> основан на переходе к~двумерному фурье-образу матрицы поля:
\begin{equation}\label{FourierDef}
    E\left(k_x, k_y, z\right)=\iint E\left(x,y,z\right)e^{-ik_xx-ik_yy}\,dxdy
\end{equation}

Первое из уравнений системы (\ref{Split}) в фурье-представлении будет выглядеть следующим образом:
\begin{equation}\label{DiffractionFourier}
    2i\frac{\partial E\left(k_x, k_y, z\right)}{\partial z}= (-k_x^2-k_y^2)E\left(k_x, k_y, z\right)
\end{equation}
Его решение определяется формулой
\begin{equation}\label{DiffractionFourierSolve}
    E\left(k_x, k_y, z+\Delta z\right)= E\left(k_x, k_y, z\right)\exp\left\{\dfrac{i}{2}(k_x^2+k_y^2)\left|E\left(k_x, k_y, z\right)\right|^2\right\}
\end{equation}

Для выполнения быстрого преобразования Фурье использовалась свободно распространяемая библиотека FFTW (версии 2.3) \cite{FFTW}.
Подробнее c реализацией и методом создания алгоритма для FFTW можно ознакомиться в статьях \cite{FFTW2_Generator_99, FFTW1_98}.
Данная реализация БПФ предполагает ленточное распределение матрицы поля по процессам.
Кроме того, FFTW, как любое быстрое преобразование Фурье, эффективнее работает на матрицах, размеры которых являются степенями двойки.

Рассмотрим алгоритм параллельного двумерного преобразования Фурье.
Вначале выполняется быстрое фурье-преобразование по строкам.
Этот этап происходит локально на каждом процессе, поскольку процесс содержит в своей оперативной памяти всю строку матрицы.
Затем происходит транспонирование распределённой матрицы, что связано с обменами данными между всеми процессами (то есть каждый обменивается с каждым).
Далее снова выполняется быстрое фурье-преобразование по строкам, которые до транспонирования являлись столбцами.

Существенной особенностью алгоритмов БПФ является расположение полученных коэффициентов в памяти процессоров.
Для преобразования Фурье естественным является транспонированное расположение результата в памяти всех процессов.
Этому соответствует ключ \\ FFTW\_TRANSPOSED\_ORDER функции, реализующей преобразование Фурье.
Его альтернативой является ключ FFTW\_NORMAL\_ORDER, при задании которого после выполнения преобразования Фурье
проводится дополнительное транспонирование матрицы спектра.
Кроме того существует параметр указанной функции, позволяющий использование дополнительного временного массива для ускорения преобразования.
Наконец, при создании плана фурье-преобразования существует возможность оптимизировать план с целью ускорения работы функции.
Это достигается использованием ключей FFTW\_ESTIMATE (грубая оценка) и FFTW\_MEASURE (при~этом производятся замеры времени пересылок, выполняемых в фурье-преобразовании и их оптимизация).
Как показали тесты, применение этих ключей обосновано для последовательного преобразования,
тогда как для параллельной версии различие скорости расчётов при использовании и без использования этого ключа отличаются в пределах статистической ошибки.

Учёт нелинейности при использовании данного метода осуществляется в рамках описанного в разделе 3. расщепления по физическим факторам.
