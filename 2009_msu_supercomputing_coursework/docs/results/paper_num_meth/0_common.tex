\vspace{0.5em}
\noindent \textbf{1. Введение и постановка задачи.}
\vspace{0.5em}

Нелинейное уравнение Шрёдингера играет существенную роль во многих областях физики, а также химии, экономики и иных наук.
Это уравнение на комплексную функцию $A(\vec{r},t)$ имеет вид:

\begin{equation}\label{NonlinearShredinger}
    \alpha i \dfrac{\partial A}{\partial t} = \beta \dfrac{\partial^2 A}{\partial x^2} + \gamma \left|A\right|^2 A
\end{equation}

Такое название присвоено уравнению (\ref{NonlinearShredinger}) потому, что его линейная часть совпадает с уравнением Шрёдингера.
В оптических приложениях чаще используется наименование <<нелинейное уравнение квазиоптики>>.

В нелинейной оптике это уравнение описывает самофокусировку  светового пучка в среде с кубической нелинейностью.
При описании распространения мощных коротких лазерных импульсов в диспергирующих средах
нелинейное уравнение для медленно меняющейся комплексной огибающей светового поля дополняют членами,
учитывающими дисперсию высших порядков, нелинейность отклика самоиндуцированной плазмы, потери на ионизацию и поглощение лазерного излучения в веществе \cite{KandidovShlenovKosarevaReview2009}.

В теории лазеров оптического диапазона с помощью нелинейного уравнения квазиоптики изучается усиление излучения, поиск собственных частот и типов колебаний поля в резонаторе.
Для исследования динамики усиленного спонтанного излучения в рентгеновском лазере используется квазиоптическое уравнение для поперечной корреляционной функции поля излучения \cite{LadaginStarikov1998}.

Распространение лазерных импульсов в оптических волноводах также описывается нелинейным уравнением квазиоптики для комплексной огибающей светового поля,
частными решениями которого являются темные (в области нормальной дисперсии групповых скоростей) и светлые (в области аномальной дисперсии групповых скоростей) солитоны \cite{Agrawal2001,Mahankov1983,VitkovskiyFedoruk2008}.

В физике сверхнизких температур нелинейное уравнение Шрёдингера используется для описания поведения неидеального бозе-газа со слабым взаимодействием между частицами \cite{Kadomcev1997}.
В работе \cite{Belyaeva2005} обращается внимание на математическую аналогию между теорией солитонных волн материи и теорией оптических солитонов в волоконных световодах.
Также нелинейное уравнение Шрёдингера используется в ядерной физике в рамках квантово-гидродинамической модели \cite{Kartvenko1993}, и, собственно, в гидродинамике для описания волн на поверхности жидкости \cite{Zeytunyan1995}.

В данной статье будет рассматриваться нелинейное уравнение квазиоптики в применении к задаче самофокусировки лазерного пучка.
Эффект самофокусировки возникает под действием изменения показателя преломления, вызванного Керровским эффектом.
В этом случае уравнение~(\ref{NonlinearShredinger}) с начальными условиями для гауссового пучка записывается в виде:
\begin{equation}\label{MainDim}
    \left\{
    \begin{array}{rcl}
        2ik\dfrac{\partial E}{\partial z} & = & \dfrac{\partial^2 E}{\partial x^2} + \dfrac{\partial^2 E}{\partial y^2} +
        \dfrac{2k^2}{n_0}n_2\left|E\left(x,y,z\right)\right|^2E\left(x,y,z\right)\\
        \\
        E(x,y,0) & = & E_0\exp\left\{-\dfrac{x^2+y^2}{2a_0^2}\right\},\quad (x,y)\in[-l,l]^2
    \end{array}
    \right.
\end{equation}

Здесь $E\left(x,y,z\right)$ "--- напряжённость электрического поля, $k$ "--- волновое число, $z$ "--- координата вдоль оси распространения светового пучка,
$x,y$ "--- координаты в поперечном сечении, $n_0$ и $n_2$ "--- линейный и нелинейный коэффициенты преломления.
В данном уравнении учтены такие физические факторы, как дифракция лазерного пучка и кубическая (по полю) нелинейность.

После обезразмеривания входящих в уравнение величин на их характерные значения $E=\tilde{E}\cdot E_0$, $x=\tilde{x}\cdot a_0$, $y=\tilde{y}\cdot a_0$, $z=\tilde{z}\cdot ka_0^2$ система будет выглядеть следующим образом:
\begin{equation}\label{MainNoDim}
    \left\{
    \begin{array}{rcl}
        2i\dfrac{\partial \tilde{E}}{\partial \tilde{z}} & = & \Delta_{\perp}\tilde{E} + R\left|\tilde{E}\right|^2\tilde{E}\\
        \\
        \tilde{E}(\tilde{x},\tilde{y},0) & = & \exp\left\{-\dfrac{\tilde{x}^2+\tilde{y}^2}{2}\right\}, \quad (x,y)\in\left[-\dfrac{l}{a_0},\dfrac{l}{a_0}\right]^2
    \end{array}
    \right.
\end{equation}


Здесь введены обозначения для поперечной части лапласиана $\Delta_{\perp}$ и коэффициента нелинейности $R$, которые будут использоватсья в дальшнейшем:
\begin{equation}
    \Delta_{\perp} = \dfrac{\partial^2}{\partial x^2} + \dfrac{\partial^2}{\partial y^2}
\end{equation}
\begin{equation}
    R = \dfrac{2k^2}{n_0} n_2 E_0^2 a_0^2
\end{equation}


\vspace{1em}
\noindent \textbf{2. Причины необходимости использования параллельных методов решения.}
\vspace{0.5em}

Основные проблемы численного моделирования задачи филаментации лазерных импульсов связаны с многомасшабностью задачи.
Поперечные масштабы пучка примерно на два порядка превосходят возникающие в нем структуры.
В то же время размер расчётной сетки должен на порядок превосходить радиус пучка, чтобы границы сетки не отсекали существенные части пучка,
а также чтобы иметь некоторую <<буферную область>>, в которую могла бы расширяться низкоинтенсивная периферийная часть пучка,
которая существенно влияет на распространение филамента \cite{KandidovShlenovKosarevaReview2009}.
В~противном случае также неизбежно возникновение краевых эффектов, приводящих к искажению решения.
На диаметр филамента должно приходиться достаточное количество точек (не менее 10), иначе резкие перепады интенсивности в окрестности филамента будут содержать слишком высокие пространственные частоты, что приведёт к невыполнению критерия Найквиста, наложению частот и, как следствие, неадекватности получаемого решения.
Как показывает практика, этот фактор является важным не только для метода решения, основанного на преобразовании Фурье, но и для остальных методов.

Таким образом, количество точек в поперечном сечении может достигать $10^4$ по каждой поперечной координате.
Рассматриваемая в статье задача не имеет временной зависимости, однако в реальных задачах филаментации рассматриваются короткие лазерные импульсы.
Для них количество временных слоев должно быть порядка $10^2-10^3$, а значит общее количество точек достигает величины порядка $10^{11}$,
а потребность в оперативной памяти "--- величины около 100~Гб.


\vspace{1em}
\noindent \textbf{3. Параллельные алгоритмы решения задачи.}
\vspace{0.5em}

Нами рассматриваются три метода численного решения уравнения (\ref{MainNoDim}).
Первый из них основан на использовании явной разностной схемы. Два других предусматривают предварительное расщепление по физическим факторам, при котором
интегрирование нелинейного уравнения квазиоптики сводится к последовательному интегрированию на каждом шаге интегрирования двух уравнений,
первое из которых описывает только дифракцию,а  второе "--- только нелинейность. Эти уравнения имеют следующий вид:
\begin{equation}\label{Split}
    \left\{
    \begin{array}{rcl}
        2i\dfrac{\partial E}{\partial z} & = & \Delta_{\perp}E \\
        \\
        2i\dfrac{\partial E}{\partial z} & = & R\left|E\right|^2E
    \end{array}
    \right.
\end{equation}

Если считать интенсивность поля ($I \sim |E|^2$) неизменной на протяжении одного шага нелинейности, что соблюдается при маленьких шагах по $z$,
то интегрирование уравнения для нелинейности не представляет проблем:
\begin{equation}\label{KerrSolution}
    E(x,y,z_i + \Delta z) = E(x,y,z_i)\exp\left(-\frac{iR}{2}\left|E(x,y,z_i)\right|^2\Delta z\right)
\end{equation}

Отметим, что поскольку нелинейность является локальной, то есть набег фазы
в точке поперечного сечения зависит только от значения интенсивности поля в этой же точке,
то для решения второго уравнения из системы (\ref{Split}) можно успешно применить метод геометрического параллелизма,
который будет иметь идеальную масштабируемость при любом количестве используемых для вычислений процессоров.

Для интегрирования уравнения дифракции в случае расщепления по физическим факторам были использованы два метода:
метод на основе неявной разностной схемы и метод, использующий быстрое преобразование Фурье.

