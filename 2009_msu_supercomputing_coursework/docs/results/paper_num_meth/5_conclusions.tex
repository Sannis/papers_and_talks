\vspace{1em}
\noindent \textbf{5. Выводы.}
\vspace{0.5em}

В рамках работы было проведено исследование различных алгоритмов решения нелинейного уравнения квазиоптики (нелинейного уравнения Шредингера).
Рассматривались методы с использованием разностных схем и метод с применением преобразования Фурье.

Тестовые эксперименты показали, что при использовании схемы предиктор-корректор (Рунге-Кутта 4-го порядка)
из-за наличия поперечных координат и участия в уравнении производной по ним необходимо
использовать слишком маленький шаг по координате z для получения устойчивого решения.
Таким образом, эту схему имеет смысл использовать в расчётах, где величина шага уже лимитирована
другими особенностями поставленной задачи. Также следует отметить, что для реализации этого алгоритма
необходимо держать в памяти 3 дополнительные матрицы равные по размеру основной,
а на каждом шаге необходимо 4 раза обмениваться граничными значениями локальных блоков матрицы.

В случае применения неявной консервативной схемы, устойчивой при любом шаге интегрирования,
шаг необходимо выбирать основываясь на эмпирических физических оценках.
Кроме того, для реализации этого алгоритма требуется только одна дополнительная матрица,
равная основной (она используется для хранения коэффициентов метода прогонки).
Алгоритм показал отличную масштабируемость на кластере IBM Bluegene/P,
которая не сильно пострадала даже в случае периодического сохранения данных вычислений на диск.
Это объясняется небольшой тактовой частотой процессоров при наличии быстрой сети и применении MPI Parallel I/O.

При использовании Фурье-метода можно обойтись без использования дополнительной матрицы,
тем самым по сравнению с методом Рунге-Кутта увеличить размер расчётной сетки в 2 раза
по каждой координате при использовании того же количества памяти.
Из-за более сложной организации пересылок при расчёте параллельного Фурье-преобразования
по сравнению с остальными алгоритмами его масштабируемость ниже.
Также наблюдается провал в производительности при небольших размерах матриц.
Однако скорость вычислений для этого метода больше, по крайней мере при использовании до 64 процессов,
что для применения на СКИФ МГУ <<Чебышёв>> делает его более удачным.

Таким образом, нельзя чётко сказать, какой метод является лучше в общем случае.
В случае, если задача не накладывает каких-то особых ограничений, лучше использовать Фурье-метод.
Если же имеется возможность использовать для расчёта очень большое число процессоров,
то целесообразно применения метода с неявной разностной схемой. Также этот метод будет применим
для задач с неравномерной сеткой по поперечному сечению, так как для них нет алгоритма
быстрого преобразования Фурье и метод с его использованием теряет свою актуальность.

Коллектив авторов выражает благодарность администрации кластеров СКИФ <<Чебышёв>> НИВЦ МГУ и IBM Bluegene/P ВМиК МГУ
за предоставленное процессорное время для проведения тестирования алгоритмов.
