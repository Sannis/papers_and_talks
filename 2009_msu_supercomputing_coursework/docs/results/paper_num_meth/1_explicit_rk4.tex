\vspace{1em}
\textbf{3.1. Метод на основе явной разностной схемы.}
\vspace{0.5em}

Уравнение (\ref{MainNoDim}) можно переписать в виде:
\begin{equation}
    \begin{aligned}
        \Delta E(x, y, z_i)=E(x, y, z_i+\Delta z)-E(x, y, z_i)=\frac{\partial E(x, y, z_i)}{\partial z}\Delta z = \\
        = \frac{1}{2i}(\Delta_{\perp}E(x, y, z_i) + R |E(x, y, z_i)|^2 E(x, y, z_i)) \Delta z = f(x, y, z_i)\Delta z
    \end{aligned}
\end{equation}

Для его решения не достаточно применять метод Эйлера, а нужно использовать более устойчивые методы из класса <<предиктор-корректор>>,
например, метод Рунге-Кутты 4-го порядка. К~сожалению, применение метода Рунге-Кутты ограничено
критерием Куранта-Фридрихса-Леви, по которому шаг по оси $z$ должен быть меньше чем $c \times \max(\Delta x,\Delta y)^{2}$,
что при увеличении количества точек в поперечном сечении приводит к чрезвычайно малому шагу по $z$.
Метод Эйлера, соответственно, требует ещё меньшего шага для получения сколько-нибудь реалистичного решения.

Кроме того, для реализации простейшей схемы расчёта $n$-го порядка точности с помощью явной схемы
необходимо на каждом шаге рассчитать $n$ матриц размером в поперечную сетку.
Это требует очень больших затрат памяти, что является минусом этого алгоритма.

Возможным решением проблемы расходования памяти и большого количества пересылок является использование коэффициентов, отличных от классических в формулах Рунге-Кутты,
например рассчитанных на основе квадратурных формул Радо и Лобатто\cite{RK_Rado_Lobatto},
которые дают аналогичный порядок точности при использовании меньшего количество ступеней в методе Рунге-Кутты.
При использовании этого метода стоит обратить внимание на правильную организацию пересылок данных
при синхронизации промежуточных результатов между процессорами на каждой ступени алгоритма.
Подробнее про возможность использования параллельного метода Рунге-Кутты можно ознакомиться в \cite{RK_Parallel_Houwen_2001, RK_Parallel_Jackson_2001}
