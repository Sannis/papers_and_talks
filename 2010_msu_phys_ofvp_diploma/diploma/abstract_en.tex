\documentclass[12pt,a4paper]{article}

\usepackage{diploma}

\hypersetup{
    pdftitle={Diploma thesis abstract},
    pdfsubject={Self-focusing of laser beams with a regular transverse structure and comparative analysis of pulses filamentation at wavelengths 0.8 and 10 microns in the air},
    pdfauthor={Oleg Efimov},
    pdfkeywords={diploma, filamentation, laser, physics, optics, numerical methods, parallel programming, self-focusing, diffraction, dispersion, Schrödinger equation}
}

\begin{document}

\newgeometry{left=1.5cm,right=1.5cm,top=2.5cm,bottom=2.5cm}

\thispagestyle{empty}

\begin{center}
{\Large Diploma thesis abstract} \\

\vspace{2ex}

{\large Self-focusing of laser beams with a regular transverse structure and~comparative analysis of pulses filamentation at wavelengths 0.8~and~10~microns in the air} \\

\vspace{2ex}

{\large Efimov\,O.\,V.} \\
\end{center}

\vspace{2ex}

The process of filaments genesis in the beams with a regular transverse structure having the~form
of in-phase and out-of-phase amplitude perturbations in a medium with Kerr nonlinearity is numerically investigated.
When the beam power is 6.6 times greater than the critical power of Gaussian beam self-focusing,
in-phase modulation brings to one filament while out-of-phase modulation results in 4 filaments of 1.65 critical powers each.
In the case of in-phase modulation filamentation length is 2.3 times less than in the case of out-of-phase modulation.
The influence of random perturbations with a Gaussian spectral correlation function is simulated.
It is shown that the initial pulse amplitude modulation with a relative fluctuation of up to 0.5
doesn't appreciably affect the filaments formation pattern, but only slightly (up to 5--10\%) increases the filamentation length.

A program, which allows numerical investigation of filamentation in case of axial-symmetric sub-picosecond pulses
at the CO2 laser wavelength, is developed. A comparative analysis of the~filamentation of laser pulses at wavelengths
of 800 nanometers and 10 microns in the air is carried out, taking into account the ionization of nitrogen and oxygen molecules in the air.
Parameters of the filament and the plasma channel for these wavelengths are obtained for the pulse power 1.5 times greater than the critical power of self-focusing.
It is shown that for the same exceeds ща the critical power and equal initial intensity the pulse at a wavelength of 10 microns causes
a more wide plasma channel with a lower density of ions, than at the wavelength of 800 nm. Filamentation distance isn't much changes.

Appendices to the diploma describe conservative differentially-difference schemes for nonlinear parabolic quasioptics equation
for both one-dimensional and axial-symmetric cases. A comparative analysis of parallel numerical algorithms for solving quasioptics equation,
namely the method based on the differentially-difference schemes and the Fourier method is performed.
Their performance is~studied on the SKIF MSU "Chebyshev"\ supercomputer.

\vspace{4ex}

The scientific adviser \\
сandidate of phys.-math. sciences \hfill docent S.\,A.\,Shlenov.

\end{document}
