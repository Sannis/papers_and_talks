\documentclass[12pt,a4paper]{article}

\usepackage{diploma}

\hypersetup{
    pdftitle={Аннотация дипломной работы}
}

\begin{document}

\newgeometry{left=1.5cm,right=1.5cm,top=2.5cm,bottom=2.5cm}

\thispagestyle{empty}

\begin{center}
{\Large Аннотация дипломной работы} \\

\vspace{2ex}

{\large Самофокусировка лазерных импульсов с регулярной поперечной структурой и~сравнительный анализ филаментации на длинах волн 0.8~и~10~мкм в воздухе} \\

\vspace{2ex}

{\large Ефимов\,О.\,В.} \\
\end{center}

\vspace{2ex}

Численно исследован процесс начала формирования филаментов в пучках с регулярной поперечной структурой
в виде синфазных и противофазных амплитудных возмущений в среде с~керровской нелинейностью.
Показано, что при мощности в 6.6 раз большей по сравнению с~критической мощностью самофокусировки гауссовского пучка в случае синфазной модуляции возникает один филамент,
а в случае противофазной модуляции возникают четыре филамента, каждый из которых содержит 1.65 критических мощностей.
При этом в случае синфазной модуляции филамент возникает на расстоянии в 2.3 раза меньшей, чем в случае противофазной модуляции.
Рассмотрено влияние случайных возмущений с~гауссовой спектральной корреляционной функцией и показано,
что при амплитудной модуляции исходного импульса с относительной флуктуацией до 0.5 эти возмущения
принципиально не влияют на характер формирования филаментов, а только незначительно (до 5--10\%) увеличивают расстояние филаментации.

Разработана программа, позволяющая исследовать филаментацию аксиально"=симметричных субпикосекундных импульсов
на длине волны CO2 лазера. Проведён сравнительный анализ филаментации лазерных импульсов на длинах волн 800~нм и~10~мкм
в воздухе с учётом ионизации молекул азота и кислорода. Получены параметры филамента и плазменного канала для этих длин волн
при превышении критической мощности самофокусировки в 1.5 раза. Показано, что при одинаковом превышении критической мощности
и равных начальных интенсивностях при распространении импульса излучения с длиной волны 10 мкм возникает более широкий плазменный канал с меньшей плотностью ионов,
чем при использовании излучения с~длиной волны 800 нм. Расстояние филаментации при этом практически не изменяется.

В приложении к диплому описаны консервативные дифференциально"=разностные схемы решения нелинейного параболического уравнения квазиоптики
в одномерном и аксиально"=симметричном случаях. Проведён сравнительный анализ
параллельных численных алгоритмов решения уравнения квазиоптики, а именно метода на основе разностных схем и Фурье-метода.
Исследована их эффективность при выполнении на суперкомпьютере СКИФ МГУ <<Чебышёв>>.

\vspace{4ex}

Научный руководитель \\
к.\,ф.-м.\,н. \hfill доцент С.\,А.\,Шлёнов.

\end{document}
