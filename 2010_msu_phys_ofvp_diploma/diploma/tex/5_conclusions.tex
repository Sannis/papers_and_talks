\cleardoublepage
\phantomsection
\section*{Заключение}
\addcontentsline{toc}{section}{Заключение}

В работе рассмотрена возможность применения пучков с регулярной поперечной структурой
для управления филаментацией лазерного импульса и проведено сравнение параметров филаментов,
образующихся при самофокусировке импульсов на длинах волн 800 нм и 10 мкм.


В частности, исследована филаментация гауссового пучка с регулярной поперечной структурой
и различными фазовыми соотношениями между соседними элементами этой структуры.

Показано, что при одинаковых остальных параметрах в случае синфазной модуляции
возникает один филамент, а в случае противофазной модуляции
возникают четыре филамента, разбегающиеся от оси распространения исходного
импульса. Величина углового расхождения филаментов определяется линейной дифракцией
пучка на~начальном этапе распространения. Обнаружено, что в случае противофазного
пучка критическая мощность самофокусировки примерно в 4 раза больше, чем в синфазном случае.
Это связано с тем, что в этом случае исходный импульс образует четыре
филамента. Таким образом, существует область значений мощности пучка, при которой
в синфазном случае филамент образуется, а в противофазном "--- нет.
Также возможен режим филаментации, при котором критическая мощность превышается уже в каждом пичке
некоторой центральной области импульса, и в таком случае они фокусируются независимо друг от друга.


Рассмотрено влияние случайных амплитудных возмущений начального пучка на~процесс
возникновения филамента и показано, что при относительной амплитуде шума
$\beta \lesssim 0.5$ они значительно не влияют на процесс формирования филаментов.


В качестве второй задачи была рассмотрена филаментация осесимметричного импульса и разработана расчётная программа, учитывающая дифракцию,
керровскую нелинейность, дисперсию второго порядка и ионизацию с учётом раздельной концентрации ионов кислорода и азота в воздухе.
Показана необходимость учёта дисперсии на этапе распространения филамента. В ходе расчётов также было получено, что для исследования распространения
сформировавшегося филамента необходимо учитывать дисперсию высших порядков и, особенно, операторы волновой нестационарности.


Проведено сравнение получающихся результатов для длины волны 800~нм с известными из литературы и показано их количественное совпадение.
Для двух импульсов с~одинаковыми длительностями, дифракционной длиной и превышением пиковой мощности над критической
при длинах волн 800~нм и 10 мкм проведено сравнение характеристик возникающего филамента и плазменного канала.
Показано, что при увеличении длины волны излучения диаметр филамента и плазменного канала увеличиваются,
тогда как концентрация плазмы уменьшается.
Это поведение предсказывается аналитическими зависимостями и экспериментальными данными,
полученными для диапазона длин волн 248--1240~нм,
что не позволяют получить количественного соответствия теоретического значения
с полученным в эксперименте для длины волны 10~мкм.
Показано, что при изменении длины волны излучения увеличивается процентное содержание ионов кислорода в образующейся плазме.


Полученные результаты могут служить основой для продолжения изучения филаментации излучения $CO_2$-лазера среднего ИК-диапазона.
Эта область длин волн на~данный момент мало исследована и может нести в себе новые интересные физические эффекты и практические приложения.


Для проведения вычислительных экспериментов был написан комплект программ
для расчёта стационарной и нестационарной задачи распространения лазерного импульса в воздухе.
В связи с тем, что часть расчётов проводилась на суперкомпьютере СКИФ МГУ <<Чебышёв>> и вычислительном кластере МЛЦ МГУ,
в процессе написания этих программ были разработаны методы распараллеливания задачи
для проведения расчётов на вычислительных кластерах и проведён сравнительный анализ параллельных численных алгоритмов решения уравнения
квазиоптики. Показано, что алгоритм на основе разностных схем имеет лучшие показатели эффективности
и масштабируемости, чем метод на основе преобразования Фурье.


